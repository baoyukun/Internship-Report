%# -*- coding: utf-8-unix -*-
% !TEX program = xelatex
% !TEX root = ../thesis.tex
% !TEX encoding = UTF-8 Unicode
\chapter{从 {\CJKLaTeX} 转向 \texorpdfstring{\XeTeX}{XeTeX}}
\label{chap:whydvipdfm}

\hspace{2em}我习惯把v0.2a使用dvipdfmx编译的硕士学位论文模板称为“ \CJKLaTeX 模板”,而这个使用 \XeTeX 引擎(xelatex程序)处理的模板则被称为“{\XeTeX/\LaTeX}模板”。
从 \CJKLaTeX 模板迁移到{\XeTeX\LaTeX}模板的好处有下:
\begin{enumerate}
\item[\large\smiley] 搭建 \XeTeX 环境比搭建 \CJKLaTeX 环境更容易;
\item[\large\smiley] 更简单的字体控制;
\item[\large\smiley] 完美支持PDF/EPS/PNG/JPG图片,不需要“bound box(.bb)”文件;
\item[\large\smiley] 支持OpenType字体的复杂字型变化功能;
\end{enumerate}

\hspace{2em}当然,这也是有代价的。由于 \XeTeX 比较新,在我看来,使用 \XeTeX 模板所必须付出的代价是:

\begin{enumerate}
\item[\large\frownie] 必须把你“古老的” \TeX 系统更新为较新的版本。TeXLive 2012和CTeX 2.9.2能够编译这份模板,而更早的版本则无能为力。
\item[\large\frownie] 需要花一些时间把你在老模板上的工作迁移到新模板上。
\end{enumerate}

